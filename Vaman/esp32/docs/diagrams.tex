\documentclass{article}
\usepackage{multirow}
\usepackage{listings}
\usepackage{./karnaugh-map}
\usepackage{titlesec}
\usepackage{amsmath}
\usepackage{capt-of}
\usepackage{circuitikz}
\usetikzlibrary{shapes.geometric}
\title{Implementation of Boolean Logic in ESP32}
\date{April 2023}
\author{Mekala Patrick Manohar\\ patrickmanohar152001@gmail.com\\ FWC22119 IITH-Future Wireless Communications Assignment}
\begin{document}
\maketitle
	\tableofcontents

\section{Problem} 
		(Gate EC-2021)\\
Q.31. The propogation delays of the XOR gate, AND gate and multiplexer (MUX) in the circuit shown in the figure are 4 ns, 2 ns and 1 ns, respectively.\\
\include{figs/diagram.tex}
If all the inputs P, Q, R, S and T are applied simultaneously and held constant, the maximum propogation delay of the circuit is
\begin{enumerate}
	\item 3 ns \item 5 ns \item 6 ns \item 7 ns
\end{enumerate}
\section{Introduction}
		In the given circuit, the output of first multiplexer can be considered as the input to the second multiplexer so that the second multiplexer output can be verified using an led with the inputs being P, Q, R, S, T and output being Y.
\section{Components}
		\captionof{table}{Table1}
\label{table:1}
\begin{tabular}{|p{5cm}|p{3cm}|p{2cm}|}
	\hline
	\multicolumn{3}{|c|}{COMPONENTS}\\
	\hline
	Component& Value& Quantity\\
	\hline
	UART &  & 1\\
	\hline
	Vaman& esp32& 1\\
	\hline                        
	\hline                                    
	Jumper Wires& M-M& 20\\                  
	\hline                                    
	Breadboard&  & 1\\                         
	\hline                                     
\end{tabular}

\section{Hardware}
		Make the connections in Arduino by giving the inputs P,Q,R,S,T to observe the output Y which is verfied using led.
\section{Software}
		\begin{enumerate}
			\item In the truth table in Table4, P,Q,R,S,T are the inputs and Y is the output.
				\include{tables/table4.tex}
			\item The k map for this truth table will be a five variablek map. So, two k maps can be drawn with one map having one input variable as zero and the other k map having that input variable as one as shown below.
				\include{figs/kmap.tex}
			\item Since, the output of the mux is either 0 or 1, this output of mux i.e, Y can be verified using led.
			\item The boolean expression for the output (Y) of the second mux with the inputs (P,Q,R,S,T) will be simplified as \ref{eq:1}
				\begin{align}
					{Y} &= {RS(T'+PQ)}
					\label{eq:1}
				\end{align}
				\item The code below realizes the Boolean logic for A with y being the input to A.\\
					\fbox{\parbox{10cm}
				{//Declaring and initializing all variables as integers\\ \\
		int P,Q,R,S,T,Y;\\ \\
		//the setup function runs once when you press reset or power the board\\ \\
		void setup()$\{$ \\
		pinMode(12,INPUT);\\
		pinMode(13,INPUT);\\
		pinMode(14,INPUT);\\
		pinMode(15,INPUT);\\
		pinMode(16,INPUT);\\
		pinMode(17,OUTPUT);\\
		$\}$ \\ \\
		//the loop function runs over and over again forever\\ \\
		void loop()$\{$ \\
			P=digitalRead(12);\\
			Q=digitalRead(13);\\
			R=digitalRead(14);\\
			S=digitalRead(15);\\
			T=digitalRead(16);\\
				Y=(!T$\&\&$P$\&\&$Q)$\|$(P$\&\&$Q$\&\&$R$\&\&$S$\&\&$T);\\
			digitalWrite(17,Y);\\
		$\}$}}
	\item Execute the above code and compare the results of output theoretically and practically.
	\end{enumerate}
\end{document}
