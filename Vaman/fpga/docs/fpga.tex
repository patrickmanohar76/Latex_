\documentclass{article}
\usepackage[none]{hyphenat}
\usepackage{enumitem}
\usepackage{graphics}
\usepackage{graphicx}
\usepackage{ragged2e}
\usepackage{multirow}
\usepackage{blindtext}
\usepackage{amsmath}
\usepackage{subcaption}
\usepackage{circuitikz}
\usepackage{listings}
\usetikzlibrary{shapes.geometric}
\lstset{
  language=C++,
  basicstyle=\ttfamily\footnotesize,
  breaklines=true,
  frame=lines
  }
\title{Implementation of the below circuit using FPGA}
\date{March 2023}
\author{M Patrick Manohar\\patrickmanohar152001@gmail.com\\FWC22119\\IIT Hyderabad-Future Wireless Communication Assignment}

\begin{document}
\maketitle
  \tableofcontents

\pagebreak

\section{Problem}
  {GATE EC-2019}\\
  Q.25. In the circuit shown,the clock frequency, i.e.,the frequency of the clock signal ,is 12 KHz.The frequency of the signal at Q2 is ............ KHz.
  \begin{figure}[h]
  \centering
    \begin{tikzpicture}                                           
	\draw (2,2) rectangle (5,5);                          
	\draw (3.5,5) node[above]{$2$ $Bit$ $binary$     $counter$};                                                  
	\draw (3.3,2) -- (3.5,2.2) -- (3.7,2);              
	\draw (7,2) rectangle (10,5);                       
	\draw (8.5,5) node[above]{$Flip-Flop$};             
	\draw (8.3,2) -- (8.5,2.2) -- (8.7,2);                
	\draw (5,3) -- (5.5,3) node[above]{$MSB$} --     (6,3);                                                       
	\draw (7.25,3) node{$K$};                            
	\draw (5,4) -- (5.5,4) node[above]{$LSB$} --     (7,4);                                                       
	\draw (7.25,4) node{$J$};                             
	\draw (6.75,4) -- (6.75,3) -- (7,3);                  
	\draw (0,0) node[above]{$clock$} -- (8.5,0);         
	\draw (3.5,0) -- (3.5,2);                             
	\draw (8.5,0) -- (8.5,2);                            
	\draw (10,3) -- (11,3);                              
	\draw (9.75,4) node{$Q$} (10,4) -- (11,4);    
\end{tikzpicture}                              

    \caption{circuit}
    \label{fig:1}
  \end{figure}

\section{Introduction}
    
    The aim is to implement the above sequential circuit using D flip-flops (IC 7474) and to find out the frequency of the signal at Q2(it is given that the frequency of the clock signal is 12KHz).IC 7474 is a dual positive edge triggered D type flip flop,which means it has two separate flip-flop that are triggered by the rising edge of a clock signal.

    In the above circuit $Q_1$,$Q_2$ are inputs and $D_1$,$D_2$ are outputs.So,from the circuit the expressions of $D_1$ and $D_2$ are:

    $D_1 = Q_1'Q_2'$.\\
      $D_2 = Q_1$.\\

Below is the transition table of the above circuit which is as follows:
\pagebreak

  \begin{table}[h]
    \begin{center}
      \documentclass{article}
\usepackage{multirow}
\usepackage{capt-of}
\begin{document}
\captionof{table}{Table2}
\label{table:2}
\begin{tabular}{|p{3cm}|p{1cm}|p{1cm}|p{1cm}|p{1cm}|p{1cm}|p{1cm}|p{1cm}|}                                           
	\hline                                                
	\multicolumn{8}{|c|}{7447 - Display}\\                                                                    
	\hline                                                
	7447& $\bar{a}$ & $\bar{b}$ & $\bar{c}$ & $\bar{d}$ & $\bar{e}$ & $\bar{f}$ & $\bar{g}$\\                                                                    
	\hline                                                
	Display& a& b& c& d& e& f& g\\                                                                            
	\hline                                        
\end{tabular}
\end{document}

      \caption{Transition table}
      \label{table:2}
    \end{center}
  \end{table}

\section{Components}
  
  \begin{table}[h]
    \begin{center}
      \begin{tabular}{|c|c|c|c|c|c|c|c|c|}                         
\hline & INPUT & \multicolumn{3}{|c|}{OUTPUT} & \multicolumn{2}{|c|}{CLOCK} & Vcc & GND     \\                                                                   
\hline ARDUINO & D6 & D3 & D4 & D5 &     \multicolumn{2}{|c|}{D2} & 5V & GND \\                              
\hline 7447 & & 5 & & & 3 & 11 & 14 & 7     \\                                                                 
\hline 7474 & 5 && 9 & 2 & \multicolumn{2}{|c|}{3} & 14 & 7 \\                                          
\hline                                     
\end{tabular}                   

      \caption{Components}
      \label{table:1}
    \end{center}
  \end{table}


\section{Hardware}

  IC 7474 is a D flip-flop integrated circuit that is commonly used in digital electronics applications.It is a dual positive edge-triggered by the rising edge of a clock signal. Below is the pin diagram of IC 7474:
  \begin{figure}[h]
    \centering
    	\begin{center}
	\begin{karnaugh-map}[2][2][1][$R$][$S$]
		\minterms{1,2}
		\autoterms[0]
	\end{karnaugh-map}
	\end{center}	

    \caption{7474}
    \label{fig:2}
  \end{figure}


The connections between the esp32 and IC 7474 is as follows:
  \begin{table}[h]
    \begin{center}
      \begin{tabular}{|c|c|c|c|c|c|c|c|c|c|c|c|}
\hline  & \multicolumn{2}{|c|}{INPUT} & \multicolumn{2}{|c|}{OUTPUT} & \multicolumn{2}{|c|}{CLOCK} & \multicolumn{4}{|c|}{VCC} \\
\hline ARDUINO & D2 & D3 & D5 & D6 & \multicolumn{2}{|c|}{D13} & \multicolumn{4}{|c|}{5V} \\
\hline 7474 & 5 & 9 & 2 & 12 & 3 & 11 & 1 & 4 & 10 & 13 \\
\hline 7447 &  &  & 1 & 7 &  &  &  & 16 &  & \\
\hline
\end{tabular}

      \caption{connections}
      \label{table:3}
    \end{center}
  \end{table}


\section{Software}

The code to implement the above circuit is : \\

    \lstinputlisting{codes/helloworldfpga.v}

\end{document}
