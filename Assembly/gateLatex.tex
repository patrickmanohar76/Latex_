\documentclass{article}
\usepackage{multirow}
\usepackage{blindtext}
\usepackage{amsmath}
\usepackage{subcaption}
\usepackage{circuitikz}
\usepackage{listings}
\usepackage{./karnaugh-map}
\usetikzlibrary{shapes.geometric}

\lstset{
	language=C++,
	basicstyle=\ttfamily\footnotesize,
	breaklines=true,
	frame=lines
	}

\title{Implementation of Boolean Logic in Arduino using IC 7474}
\date{February 2023}
\author{M Patrick Manohar\\patrickmanohar152001@gmail.com\\FWC22119\\IIT Hyderabad-Future Wireless Communication Assignment}

\begin{document}
\maketitle
	\tableofcontents

\pagebreak
\section{Problem}
	(GATE EC-2022)\\
	Q.43. For the circuit shown, the clock frequency is $f0$ and the duty cycle is $25 \%$. For the signal at the $Q$ output of the Flip-Flop,
\\
	\begin{figure}[h]
		\centering
	\begin{tikzpicture}                                           
	\draw (2,2) rectangle (5,5);                          
	\draw (3.5,5) node[above]{$2$ $Bit$ $binary$     $counter$};                                                  
	\draw (3.3,2) -- (3.5,2.2) -- (3.7,2);              
	\draw (7,2) rectangle (10,5);                       
	\draw (8.5,5) node[above]{$Flip-Flop$};             
	\draw (8.3,2) -- (8.5,2.2) -- (8.7,2);                
	\draw (5,3) -- (5.5,3) node[above]{$MSB$} --     (6,3);                                                       
	\draw (7.25,3) node{$K$};                            
	\draw (5,4) -- (5.5,4) node[above]{$LSB$} --     (7,4);                                                       
	\draw (7.25,4) node{$J$};                             
	\draw (6.75,4) -- (6.75,3) -- (7,3);                  
	\draw (0,0) node[above]{$clock$} -- (8.5,0);         
	\draw (3.5,0) -- (3.5,2);                             
	\draw (8.5,0) -- (8.5,2);                            
	\draw (10,3) -- (11,3);                              
	\draw (9.75,4) node{$Q$} (10,4) -- (11,4);    
\end{tikzpicture}                              

		\caption{Circuit}
		\label{fig:1}
	\end{figure}

\begin{enumerate}
	\item frequency of $\frac{f0}{4}$ and duty cycle is 50$\%$
	\item frequency of $\frac{f0}{4}$ and duty cycle is 25$\%$
	\item frequency of $\frac{f0}{2}$ and duty cycle is 50$\%$
	\item frequency of $f0$ and duty cycle is 25$\%$ \\
\end{enumerate}

\section{Introduction}
		The Aim is to implement the above circuit in Arduino using IC 7474. IC 7474 is a dual positive-edge-triggered D-type flip-flop, which means it has two seperate flip-flop that are triggered by the rising edge of a clock signal. A 2-bit binary counter can be implemented using 2 D Flip-flops similarly a JK Flip-flop can be implemented using one D Flip-flop. Thus we will use two IC 7474 to implement the whole circuit.\\

		The LSB output of the 2-bit binary counter is given to J and K inputs of the JK Flip-flop which then gives the final Q output of the circuit. Since the inputs given to J and K are same it acts as T Flip-flop.\\
\section{Components}
	\begin{table}[h]
		\begin{center}
			\begin{tabular}{|p{5cm}|p{3cm}|p{2cm}|}
\hline
\multicolumn{3}{|c|}{COMPONENTS}\\
\hline
Component& Value& Quantity\\
\hline
Resistor& $\>$=220 Ohm& 1\\
\hline
Arduino& UNO& 1\\
\hline
Seven Segent Display& Common Anode& 1\\
\hline
Decoder& 7447& 1\\
\hline
Flip Flop& 7474& 2\\
\hline
Jumper Wires&  & 20\\
\hline
Breadboard&  & 1\\
\hline
\end{tabular}

			\caption{Components}
			\label{table:0}
		\end{center}
	\end{table}
\pagebreak
\section{Hardware}
	The IC 7474 is a type of flip-flop integrated circuit that is commonly used indigital electronics applications. It is a dual positive-edge-triggered by the rising edge of a clock signal. Below is the pin diagram of IC 7474. \\
		\begin{figure}[h]
			\centering
			\begin{center}
	\begin{karnaugh-map}[2][2][1][$R$][$S$]
		\minterms{1,2}
		\autoterms[0]
	\end{karnaugh-map}
	\end{center}	

			\caption{7474}
			\label{fig:2}
		\end{figure}

	The connections between Arduino UNO and two IC 7474 is given in below Table \\
	\begin{table}[h]
		\begin{center}
	\begin{tabular}{|c|c|c|c|c|c|c|c|c|}                         
\hline & INPUT & \multicolumn{3}{|c|}{OUTPUT} & \multicolumn{2}{|c|}{CLOCK} & Vcc & GND     \\                                                                   
\hline ARDUINO & D6 & D3 & D4 & D5 &     \multicolumn{2}{|c|}{D2} & 5V & GND \\                              
\hline 7447 & & 5 & & & 3 & 11 & 14 & 7     \\                                                                 
\hline 7474 & 5 && 9 & 2 & \multicolumn{2}{|c|}{3} & 14 & 7 \\                                          
\hline                                     
\end{tabular}                   

			\caption{Arduino - 7474}
			\label{table:1}
		\end{center}
	\end{table}

	The truth table for the circuit is given in below table \\
	
		\begin{table}[h]
		\begin{center}
			\documentclass{article}
\usepackage{multirow}
\usepackage{capt-of}
\begin{document}
\captionof{table}{Table2}
\label{table:2}
\begin{tabular}{|p{3cm}|p{1cm}|p{1cm}|p{1cm}|p{1cm}|p{1cm}|p{1cm}|p{1cm}|}                                           
	\hline                                                
	\multicolumn{8}{|c|}{7447 - Display}\\                                                                    
	\hline                                                
	7447& $\bar{a}$ & $\bar{b}$ & $\bar{c}$ & $\bar{d}$ & $\bar{e}$ & $\bar{f}$ & $\bar{g}$\\                                                                    
	\hline                                                
	Display& a& b& c& d& e& f& g\\                                                                            
	\hline                                        
\end{tabular}
\end{document}

			\caption{Truth Table}
			\label{table:2}
		\end{center}
		\end{table}

		The kmap for the circuit is \\
		\begin{figure}[h]
			\centering
			\begin{karnaugh-map}[4][2][1][$K$][$J$][$Qn$]
\minterms{2,3,4,6}
\autoterms[0]
\implicant{3}{2}
\implicantedge{4}{4}{6}{6}
\end{karnaugh-map}

			\caption{kmap}
			\label{fig:3}
		\end{figure}
\section{Software}
	The Arduino code for the given circuit using IC 7474 is \\
	\lstinputlisting{gatelatex1.asm}
\end{document}
